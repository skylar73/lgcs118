\documentclass{article}

\input{PomonaLgcsFormatting}

\title{Tariana Notes}
\author{Skylar Litz}
\date{February 19, 2021}

\begin{document}

\maketitle
\\ \\
\textbf{Research period 1:} word order of clauses and noun phrases, as well as other major features necessary to understand the basics of the language (e.g. basic morphology on nouns and verbs)


\section{Chapter 3: Word classes}
\begin{itemize}
	\item Tariana has 3 open word classes, 2 semi-closed classes, and 12 closed classes of words.
	\item Open word classes
	\begin{itemize}
		\item \textbf{Verbs:}
		\begin{itemize}
			\item can be cross-referenced for person, number, and gender
			\item valency can be increased/decreased using valency-increasing/reducing derivations (causative, passive, reciprocal(?))
			\item take a negative suffix and verbal classifiers (ch 12-13, 17)
			\item usually have tense-aspect-evidentiality, mood, and Aktionsart clausal enclitics (ch 14-16)
			\item can form serial verb constructions and complex predicates
			\item divide into two classes: can or cannot take predicates
			\item active transitive or intransitive verbs must take cross-referencing prefixes to mark A/$S_a$ constituent (see preface?). Only have 1 obligatory prefixal position (when prefixed negation \emph{ma-} is used, personal cross-referencing prefixes are omitted and person/gender/number distinctions are neutralized)
			\item stative verbs ($S_O$) verbs and verbs of physical states ($S_{iO}$) verbs don't take any cross-referencing markers (roughly corresponds to what was inherited from Proto-Arawok)
			\item prefixed verbs show predicate-argument agreement with cross-referencing prefixes, prefixless verbs don't.
			\item verbs fall into further classes depending on their transitivity (bottom of page 68, not sure if important or not). Further classes include: ambitransitive verbs, intransitive verbs, active verbs ($S_a$), stative verbs ($S_O$), verbs of physical state ($S_{iO}$).
		\end{itemize}

		\item \textbf{Nouns:}
		\begin{itemize}
			\item have grammatical categories: gender, noun classification, possession, number, case, nominal tense, and extralocality
			\item \textbf{Gender.} gender is realized in person-number cross-referencing on verbs and obligatorily possessed nouns (table 6.1, ch 5). Nouns usually bear no overt gender marking (but there are a few gender-sensitive derivational suffixes, see ch 10)
			\item \textbf{Noun classification.} noun classification is realized through classifier agreement on modifiers in head-modifier noun phrases (in english, heads of noun phrases must be a noun or a pronoun). There are different options for what modifiers can be (adjective, numeral, demonstrative, or a member of some other closed classes). For each type of modifier from a closed class, a different set of classifiers is used (ch 5 and 11). Most classifiers can be used on the noun itself (like a derivational suffix). ''Semantically, noun classification overlaps with gender to a certain extent'' (what?? I thought that nouns don't have a gender marking?)
			\item \textbf{Possession.} Nouns can be split into inalienably possessed (prefixed) and alienably possessed (prefixless) (ch 6). There are special cases for some cases (focussed A/S, topical non-A/S, locative case, instrumental-comitative) (ch 7).
			\item \textbf{Number.} Distinctions include singular, plural, unmarked (collective), and associative. Different semantic groups of nouns have different number distinctions and number marking (ch 8).
			\item \textbf{Tense.} nouns have a three-fold tense distinction (part, future, and unmarked) and can also be characterized for extralocality (the participant referred to is in a different place or is the only one to not do something) (ch 9). Enclitics such as contrastivity, coordination, and degree markers (augmentative, diminutive, and approximative) can only occur on nouns and NPs (ch 9, 10).
			\item unlike with verbs, some nominal categories including number, gender, classifier, and case can be marked more than once within a normal word (ch 4)
			\item nouns have a complex inflectional structure (ch 4). They fall into several distinct classes according to their morphological possibilities and the properties of their referents. (this classification is different genders and classifiers).
			\item Categories of nouns detailed here (pg 69) include: prefixed and non-prefixed nouns, nouns with a human referent, kinship nouns (closed subset of nouns that must have a human referent) [interesting, come back to this?], personal names ('names of blessing', eleven members in closed class), inanimate nouns, inherently locational nouns and placenames.
		\end{itemize}
		\item \textbf{Adjectives:}
		\begin{itemize}
			\item any adjective can be used without a nominal head in a noun phrase (ch 4) (e.g. inam matfi:te 'bad woman', or matfi:te 'bad one' (pg 72).
			\item adjectives can be underived or derived. Underived adjective form a closed class of around 20 items.
			\item There are no specific adjectives that refer to form (ex round, hollow, curved), but instead the meanings are expressed with the help of classifiers
			\item underived adjective groups: dimension, age, value, color, physical properties, other (love (by women)).
			\item all adjectives require classifier agreement with the head noun, number agreement is optional with nouns referring to an inanimate object, and obligatory with human and higher animates.
			\item The suffix/clitic \emph{-iha} behaves interestingly. Add stuff on this? pg 74,75
			\item all adjectives share these properties with nouns: they can be used as NP heads without special word class-changing morphology, they can be used as copula complements (see 21.4.1). unlike nouns, adjectives do not nominal tense and extralocality marking (ch 4 and 9). They cannot be heads of possessive constructions
			\item they also share some properties with stative verbs: when used as predicates they take the same morphology as stative verbs, they can be used in some imperative clauses similarly to stative verbs, underived adjectives can never take cross-referencing prefixless
			\item adjectives also have properties that are not shared with either nouns or verbs (see list on pg 76). This is typologically unusual!
		\end{itemize}
	\end{itemize}
	\item Closed classes
	\begin{itemize}
		\item Closed word classes that can be used as NP heads and as modifiers: personal pronouns, specifier articles, interrogative-distributive, demonstrative, gestural deictic, distributive individualiser, general indefinite, numerals, quantifiers
		\item adpositions are a closed class that can only be used as NP heads
		\item connectives are a closed class of 3 members, share a few properties with nouns
		\item \emph{hisada} (sg), \emph{hisaka} (pl) is in a class of its own. Occurs rarely in the predicate slot, as a part of a serial verb construction. Unusual feature: number marking pattern not found with nay other word class (ch 8) (example given on pg 80)
		\item there are some other closed classes
	\end{itemize}
	\item Table 3.7 has a nice summary of word classes and functional slots
\end{itemize}

\section{Chapter 4: Nominal morphology and noun structure}
page 475 (498)
\begin{itemize}
	\item polysynthetic language which combines head-marking morphology with elements of dependent marking.
	\item Unusual: nouns in Tariana are not only derivationally, but also inflectionally complex
	\item Nouns can take up to 16 structural positions (one is prefix) including classifiers, tense (future and past), extralocativity, contrastitivity, cases and other affixes and enclitics.
	\item theres a lot about noun structure here, including a big chart which seems useful
	\item lots of examples, probably use these


\end{itemize}


\section{Chapter 21: Clause types and other syntactic issues}
\begin{itemize}
	\item The structure of noun phrases
	\item Coordination of noun phrases
	\item Structure of predicates
	\item Types of clauses
	\item Grammatical relations
\end{itemize}

\section{Chapter 25: Discourse Organization}
pg 561 (584)
\begin{itemize}
	\item rules for ordering constituents within an NP, clause, or sentence depend on types of constituents, the construction type and the pragmatics.
	\item
\end{itemize}


\end{document}
