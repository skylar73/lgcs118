\documentclass[12pt]{article}

\input{PomLingHandoutTemplatePreamble}

\title{Introduction to Tariana}
\author{Skylar Litz}
\date{February 17, 2021}

\begin{document}

\maketitle

%\textbf{Goals of talk:} This is a handout including (mainly) word order of clauses and noun phrases, as well as other major features necessary to understand the basics of the language (e.g. basic morphology on nouns and verbs).

\section{Introduction}
 Tariana is an Arawak language spoken in the multilingual area of the Vaupés basin in northwestern Brazil. All data, examples, and generalizations presented here come from Aikhenvald (2003).
\section{Constituent Order}
  Tariana is a \textbf{`pragmatically ordered'} language so establishing a basic constituent order is not possible. In both texts and conversations, any constituent order is possible and all different organizations are used.

\subsection{Details on Pragmatic Ordering}
\begin{itemize}
  \item Sentences that are translated from the nearby language Tucano are generally \textit{verb-final} while sentences translated from Portuguese are often \textit{verb-medial}. For this reason, when studying Tariana it is important not to rely on translated material to understand constituent order.

  % pg 586 of pdf
  \item Instead of having a set constituent order, pragmatic parameters are used to order constituents in main clauses. These parameters include new vs. old information, relative topicality, definiteness, specificity, and contrast.

  \item There are also specific case markings for topicality of a non-subject constituent, specific and/or definite referents, and focussed and contrastive subjects.

  % go over these in next presentation?
  %\item Clause initial and pre-predicate positions: Topicality
  %\item Post-predicate positions: contrast and/or disambiguation
  %\item Unmarked non-subject immediately before the predicate

  % pdf pg 591
  \item There are a few constructions that maintain a fixed constituent order. These are,
  \begin{itemize}
    \item position of the copula component with respect to the copula
    \item position of interrogative words
    \item position of clause connectors
    \item position of the predicate in dependent clauses
    \item imperatives and apprehensives
    \item double S-clauses
  \end{itemize}

  \item A unique feature of Tariana compared to other Arawak languages is that in possessive constructions, the possessor always precedes the possessed noun.

\end{itemize}


\section{Order of words in Noun Phrases}
A noun phrase consists of a head plus one or more modifiers.

  \subsection{NP heads}
  \begin{itemize}
    \item NP heads can be a noun, adjective, or a member of certain closed classes (demonstratives, specifier articles, quantifier, and deictics).
    \item The head of an NP forces classifier (or animacy) agreement on any modifier that is present on the head. %[SEE CHAPTER 5] (more on this in second presentation? or add diarrhea example from page 93 (pdf pg 116))
    %\item If the NP head is animate, number agreement is also required (see 8.4)
  \end{itemize}
  \subsection{NP modifiers}
  \begin{itemize}
    \item Modifiers can be adjectives, members of certain closed classes, and some nouns.
    \item Adjectives or closed class modifiers can be used in either the prehead or posthead position, except for specifier articles, demonstratives, and the quantifier \textit{kanapada} which must always precede the NP head.
    \item The placement of modifiers prehead or posthead depends on the definiteness and specificity of the head noun. If a noun is definite or specific, modifiers tend to be placed before the noun. Indefinite or non-specific nouns usually have modifiers placed after the noun.
  \end{itemize}
  %\subsubsection{Example}
  For example (Figure \ref*{naughtytapir}), in a story being told about a well-known naughty tapir who destroys gardens, the modifier adjective `bad' is placed before the head noun, `tapir.'

\begin{figure}[h]
  \centering
  \includegraphics[scale = 0.5]{naughtytapir.png}
    \caption{(Aikhenvald 2003 page 476)}
    \label{naughtytapir}
\end{figure}

  \begin{itemize}
    \item An NP can contain only one prehead modifier, so if one modifier is required to appear before the noun, the rest will appear after.
    % pg 505
    \item In narratives and conversations, full NPs are not very frequent. Instead, headless NPs are used in which a classifier is used to identify the referent.
    %\item possessive constructions the possessor always precedes the possessed noun (different from different Arawak languages) pg 506
  \end{itemize}
For example (Figure \ref*{oneman}), the man who lives alone with his children is introduced in a headless NP with a numeral and a classifier.

\begin{figure}[h]
  \centering
  \includegraphics[scale = 0.69]{oneman.png}
    \caption{(Aikhenvald 2003 page 482)}
    \label{oneman}
\end{figure}

\end{document}
